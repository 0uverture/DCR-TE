\documentclass{article}

\usepackage{hyperref}
\usepackage[utf8]{inputenc}
\usepackage{cite}
\usepackage{placeins}
\usepackage{listings}
\usepackage{xcolor}
\usepackage{graphicx}
\usepackage{relsize}

% \linespread{1.5}

\definecolor{listinggray}{gray}{0.9}
\definecolor{lbcolor}{rgb}{0.9,0.9,0.9}

\newcommand\cpp{C\nolinebreak[4]\hspace{-.05em}\raisebox{.4ex}{\relsize{-3}{\textbf{++}}}}

\lstset {
	    backgroundcolor=\color{lbcolor},
	    tabsize=4,    
	    language=[GNU]C++,
        basicstyle=\scriptsize,
        upquote=true,
        aboveskip={1.5\baselineskip},
        columns=fixed,
        showstringspaces=false,
        extendedchars=false,
        breaklines=true,
        prebreak = \raisebox{0ex}[0ex][0ex]{\ensuremath{\hookleftarrow}},
        frame=single,
        numbers=left,
        showtabs=false,
        showspaces=false,
        showstringspaces=false,
        identifierstyle=\ttfamily,
        keywordstyle=\color[rgb]{0,0,1},
        commentstyle=\color[rgb]{0.026,0.112,0.095},
        stringstyle=\color[rgb]{0.627,0.126,0.941},
        numberstyle=\color[rgb]{0.205, 0.142, 0.73}
}

\author{Mikkel Gaub, Malthe Ettrup Kirkbro \& Mads Frederik Madsen}
\title{Thesis Agreement}
\date{\today}

\begin{document}

\maketitle
\thispagestyle{empty}

\vspace{\fill}

\begin{abstract}
Lorem ipsum
\end{abstract}

\pagebreak

\tableofcontents

\pagebreak

	\section{Introduction}

	In an increasingly digital business world, collaboration between companies is an everyday occurrence.
	An intuitive way to model such a collaboration, is by using workflow, which will be the basis of this article.
	Performing collaborative work on a workflow between multiple, possibly untrusted, parties is mostly the same challenge as is posed in any distributed network. 
	Most solutions to this are, however, cumbersome in that they require a very large amount of data to be transmitted.
	As the new innovation applied in this project is Intel SGX, it will be investigate what sort of advantages it grants in the implementation of a distributed network.

		\subsection{Goals}

		During this project we will investigate whether or not it is possible to create a secure implementation of a Dynamic Condition Response\cite{dcr-paper} (DCR) engine using Intel Software Guard Extensions\cite{intel-sgx-explained} (SGX).
		It will be split into two phases:
		\begin{enumerate}
			\item The first phase, which is described in its entirety in this report, is an investigation into the feasibility and usability of Intel SGX in conjunction with distributed DCR graphs.
			\item The second phase, will be the design and implementation of the DCR-SGX system along with a more in-depth foray into Intel SGX and its capabilities. Should this prove more easily attainable than it is currently believed to be, the implementation will be expanded to include generic applications, creating a distributed network of smart contracts\cite{smart-contracts}.
		\end{enumerate}

	\section{Related works}

	An existing platform which partially solves this problem is Ethereum\cite{ethereum-white-paper}, which is a distributed computing platform based on blockchain technologies.
	It does however also feature a currency, making computations disproportionately expensive, and the large amounts of computations required by proof-of-work based blockchains.

	\section{Background}

	The three primary components of the proposed solution are as follows: the workflow, which will be described with DCR, Intel SGX and method for achieving distributed consensus.

		\subsection{Dynamic Condition Response}

		A DCR graph is a representation of a workflow.
		The graph is made up of one or more activities with a number of relations between them. 
		The following section is loosely based on a similar description found in previous work\cite{dcreum}.

		\subsubsection{Activity}

		The activities in a DCR graph have three attributes: included, executed and pending. 
		The attributes can be true or false. 
		Furthermore an activity can have role and actor specific execution rights.

			\paragraph{Included attribute}

			If the include attribute of an activity is true, the activity is included and it can be executed. 
			If the attribute is false, the activity is excluded and can no longer be executed.

			\paragraph{Pending attribute}

			If any activity in a workflow has a pending attribute that is true and the activity is included, the workflow is in an unfinished state.
			Every time an activity is executed its pending attribute is set to false.
			This means that setting the pending attribute of an included activity to true is specifying that this activity must be executed or excluded at some point to leave the workflow in a finished state.

			\paragraph{Executed attribute}

			If an activities executed attribute is false executing the activity will set its executed attribute to true.
			Executing an already executed activity will have no effect on the executed attribute.

		\subsubsection{Relations}

		There are five types of relations which define different types of relationships between activities in a workflow. 
		These relations are:

		\begin{description}
			\item[Condition] If there is a condition relation from activity $A$ to activity $B$, then $B$ can only be executed if $A$'s executed attribute is true or $A$ is excluded.
			\item[Response] If there is a response relation from activity $A$ to activity $B$, then $B$'s pending attribute will be set to true every time $A$ is executed.
			\item[Include] If there is an include relation from activity $A$ to activity $B$, then $B$'s included attribute will be set to true every time $A$ is executed.
			\item[Exclude] If there is an exclude relation from activity $A$ to activity $B$, then $B$'s included attribute will be set to false every time $A$ is executed.
			\item[Milestone] If there is a milestone relation from activity $A$ to activity $B$, then $B$ can only be executed if $A$'s pending attribute is false or $A$'s included attribute is false.
		\end{description}

		\subsection{SGX}

		Intel SGX is a technology which allows users to define protected areas of memory, so called enclaves, and run code within these enclaves.
		Intel can then guarantee that any code run this way is unmolested by any process running outside of the enclave.
		An additional feature of Intel SGX is that the result of any code run using Intel SGX can be verified by other users, given the code and the output.
		This is made possible through asymmetric encryption, with Intel as the key distributor.
		To ensure that keys are only distributed to trusted users, Intel offers attestation services as a way of key distribution.

		\subsection{Consensus}
		Consensus is the problem of establishing agreement between two or more processes in a distributed setting. Formally consensus is defined as a protocol in which $N$ processes ($P_0, P_1, \dots, P_{n-1}$) starts in an undecided state. Each process will then propose a single value, $v_i$, from a set of values, $D$. The processes communicate their propositions to each other, and each decide on a decision-variable, $d_i$. They are now in a decided state. If the following requirements are met, the protocol solves the consensus problem:
		\begin{description}
			\item[Termination] Every process eventually sets its decision-variable. 
			\item[Agreement] The decision-variables of all correct processes are set to the same value.
			\item[Integrity] If every correct process proposes the same $v$, then all correct processes decide $d_i = v$.
		\end{description}

		In a famous result from 1985, known as the FLP impossibility result\cite{flp}, it was shown that consensus was impossible to solve in an asynchronous setting with as little as a single crashing process. The FLP impossibility result, can, very simplistically, be expressed as the problem of distinguishing a failed process from a delayed message.\\

	\section{Challenges}

		\subsection{SGX security}

		Due to SGX being a security guarantee made by Intel all work involving it, the proposed project requires absolute trust in Intel keeping private keys secure and them not violating the confidentiality of machines with their hardware.

		\subsection{Consensus}
		Given that our problem deals with agreeing on a history of a workflow, we must somehow circumvent FLP impossibility.
		A few solutions for circumventing FLP-impossibility have been suggested, for instance using a probabilistic protocol, which achieves consensus except for some negligible probability. Most notable of these is perhaps the blockchain technology, (formal analyses of the complexity and probability of success can be found in \cite{miller2014anonymous}). However, the basis of a blockchain succeeding is an incentive for peers to use computing power to secure the system. This incentive takes the form of a currency \cite{bitcoin-white-paper}. While a clever solution, we are not interested in basing our system on a currency, thus subjecting the viability of the system on the rise-and-fall of the rate and popularity of the currency. Instead we have looked at alternatives for side-stepping FLP-impossibility. We have found the survey by Correia et al. \cite{consensus-survey} to be very helpful. The following are the solutions that we have identified to have the most potential for this system:
		\begin{itemize}
			\item One solution is to assume that some subsystem or step of the protocol behaves synchronously. This is known as \textit{partial synchrony}. If the system does not live up to these assumptions, the protocol will naturally fail. The challenge is this solution to identify a subsystem that can reliably be assumed to behave synchronously. 
			\item A variation on \textit{partial synchrony} is to assume some subsystem has some "good" properties, which is otherwise not guaranteed in the environment. These subsystems are known as \textit{wormholes}. According to the survey \cite{consensus-survey}, the Trusted Platform Module (TPM) is an example of a wormhole, and since SGX can implement some of the same concepts as those available in TPM, it might be possible to use SGX as a wormhole. The survey mentions several protocols which might be transformed to solve the consensus problem, including one that uses a Unique Sequential Identifier Generator, which is essentially a monotonic counter. The challenge in this approach is to identify which of the properties in the wormhole allows side-stepping FLP impossibility, and then implementing them using SGX.
			\item The most simple way of side-stepping FLP impossibility, is simply to modify the consensus problem, i.e. weaken the requirements s.t. some failures are allowed. For instance, the PAXOS algorithm \cite{paxos} satisfy variations on the Agreement and Integrity requirements, but does not satisfy the appertaining Termination requirement \cite{fast-byz-cons}. The challenge in this approach is to identify which requirements can be most safely weakened, and how.
		\end{itemize}
		
	\section{Implementation}

	In order to demonstrate that the proposed solution could work, we demonstrate the implementation of a monotonic counter in \cpp.

	\section{Conclusion}

	\begin{thebibliography}{9}

		\bibitem{smart-contracts}
		Nick Szabo.
		\textit{Formalizing and Securing Relationships on Public Networks}.
		Peer-reviewed journal on the internet.

		\bibitem{ethereum-white-paper}
		\textit{Ethereum White Paper}.
		\url{https://github.com/ethereum/wiki/wiki/White-Paper}.
		Accessed 2017-12-11.

		\bibitem{bitcoin-white-paper}
		\textit{Bitcoin White Paper}.
		\url{https://bitcoin.org/bitcoin.pdf}.
		Accessed 2017-12-11.

		\bibitem{dcr-paper}
		Thomas T. Hildebrandt \& Raghava Rao Mukkamala.
		\textit{Declarative Event-Based Workflow as Distributed Dynamic Condition Response Graphs}.
		IT University of Copenhagen.

		\bibitem{intel-sgx-explained}
		Victor Costan \& Srinivas Devadas.
		\textit{Intel SGX Explained}.
		Computer Science and Artificial Intelligence Laboratory Massachusetts Institute of Technology.

		\bibitem{dcreum}
		Mikkel Gaub, Tróndur Høgnason, Malthe Ettrup Kirkbro \& Mads Frederik Madsen.
		\textit{Consensus in declarative process models using distributed smart-contracts}.

		\bibitem{sgx-dev-ref}		
		\textit{Intel® Software Guard Extensions SDK for Linux OS}.
		\url{https://download.01.org/intel-sgx/linux-2.0/docs/Intel_SGX_SDK_Developer_Reference_Linux_2.0_Open_Source.pdf}

		\bibitem{sgx-dev-guide}
		\textit{Intel® Software Guard Extensions Developer Guide}
		\url{https://software.intel.com/en-us/documentation/sgx-developer-guide}

		\bibitem{flp}
		Michael J. Fischer and  Nancy A. Lynch and Michael S. Paterson
		\textit{Impossibility of distributed consensus with one faulty process},
		\url{https://dl.acm.org/citation.cfm?id=214121}

		\bibitem{miller2014anonymous},
		Andrew Miller and Joseph J. LaViola, Jr.
  		\textit{Anonymous byzantine consensus from moderately-hard puzzles: A model for bitcoin},
		\url{Available on line: http://nakamotoinstitute. org/research/anonymous-byzantine-consensus}

		\bibitem{consensus-survey}
		Miguel Correia, Giuliana Santos Veronese, Nuno Ferreira Neves and Paulo Verissimo
		\textit{Byzantine consensus in asynchronous message-passing systems: a survey} 

		\bibitem{paxos}
		Leslie Lamport
		\textit{BThe Part-Time Parliament} 

		\bibitem{fast-byz-cons}
		Jean-Philippe Martin and Lorenzo Alvisi
		\textit{Fast Byzantine Consensus}



	\end{thebibliography}

\end{document}