\documentclass[12pt]{article}

\usepackage[hyphens]{url}
\usepackage{hyperref}
\usepackage[utf8]{inputenc}
\usepackage{cite}
\usepackage{placeins}
\usepackage{listings}
\usepackage{xcolor}
\usepackage{graphicx}
\usepackage{relsize}
\usepackage{newfloat}

\usepackage{times}
\usepackage[margin=2cm]{geometry}

\linespread{1.5}

\definecolor{listinggray}{gray}{0.9}
\definecolor{lbcolor}{rgb}{0.9,0.9,0.9}

\newcommand\cpp{C\nolinebreak[4]\hspace{-.05em}\raisebox{.4ex}{\relsize{-3}{\textbf{++}}}}

\DeclareFloatingEnvironment[
    fileext=los,
    listname=List of code snippets,
    name=Snippet,
    placement=tbhp,
    within=none,
]{snippet}

\lstset {
	    backgroundcolor=\color{lbcolor},
	    tabsize=4,    
	    language=[GNU]C++,
        basicstyle=\scriptsize,
        upquote=true,
        aboveskip={1.5\baselineskip},
        columns=fixed,
        showstringspaces=false,
        extendedchars=false,
        breaklines=true,
        prebreak = \raisebox{0ex}[0ex][0ex]{\ensuremath{\hookleftarrow}},
        frame=single,
        numbers=left,
        showtabs=false,
        showspaces=false,
        showstringspaces=false,
        identifierstyle=\ttfamily,
        keywordstyle=\color[rgb]{0,0,1},
        commentstyle=\color[rgb]{0.026,0.112,0.095},
        stringstyle=\color[rgb]{0.627,0.126,0.941},
        numberstyle=\color[rgb]{0.205, 0.142, 0.73}
}

\author{Mikkel Gaub, Malthe Ettrup Kirkbro \& Mads Frederik Madsen}
\title{Thesis Agreement}
\date{\today}

\begin{document}

\maketitle
\thispagestyle{empty}

\pagebreak

\tableofcontents

\pagebreak

	\section{Introduction}

	In an increasingly digital business world, collaboration between companies is an everyday occurrence.
	An intuitive way to model such a collaboration, is by using workflow, which will be the basis of this article.
	Performing collaborative work on a workflow between multiple, possibly untrusted, parties is mostly the same challenge as is posed in any distributed network. 
	Most solutions to this are, however, cumbersome in that they require a very large amount of data to be transmitted.
	As the new innovation applied in this project is Intel SGX, it will be investigate what sort of advantages it grants in the implementation of a distributed network.

		\subsection{Goals}

		During this project we will investigate whether or not it is possible to create a secure implementation of a Dynamic Condition Response\cite{dcr-paper} (DCR) engine using Intel Software Guard Extensions\cite{intel-sgx-explained} (SGX).
		It will be split into two phases:
		\begin{enumerate}
			\item The first phase, which is described in its entirety in this report, is an investigation into the feasibility and usability of Intel SGX in conjunction with distributed DCR graphs.
			\item The second phase, will be the design and implementation of the DCR-SGX system along with a more in-depth foray into Intel SGX and its capabilities. Should this prove more easily attainable than it is currently believed to be, the implementation will be expanded to include generic applications, creating a distributed network of smart contracts\cite{smart-contracts}.
		\end{enumerate}

	\section{Related works}

	An existing platform which partially solves this problem is Ethereum\cite{ethereum-white-paper}, which is a distributed computing platform based on blockchain technologies.
	It does however also feature a currency, making computations disproportionately expensive, and the large amounts of computations required by proof-of-work based blockchains.

	\section{Background}

	The three primary components of the proposed solution are as follows: the workflow, which will be described with DCR, Intel SGX and method for achieving distributed consensus.

		% \subsection{Dynamic Condition Response}

		% A DCR graph is a representation of a workflow.
		% The graph is made up of one or more activities with a number of relations between them. 
		% The following section is loosely based on a similar description found in previous work\cite{dcreum}.

		% \subsubsection{Activity}

		% The activities in a DCR graph have three attributes: included, executed and pending. 
		% The attributes can be true or false. 
		% Furthermore an activity can have role and actor specific execution rights.

		% 	\paragraph{Included attribute}

		% 	If the include attribute of an activity is true, the activity is included and it can be executed. 
		% 	If the attribute is false, the activity is excluded and can no longer be executed.

		% 	\paragraph{Pending attribute}

		% 	If any activity in a workflow has a pending attribute that is true and the activity is included, the workflow is in an unfinished state.
		% 	Every time an activity is executed its pending attribute is set to false.
		% 	This means that setting the pending attribute of an included activity to true is specifying that this activity must be executed or excluded at some point to leave the workflow in a finished state.

		% 	\paragraph{Executed attribute}

		% 	If an activities executed attribute is false executing the activity will set its executed attribute to true.
		% 	Executing an already executed activity will have no effect on the executed attribute.

		% \subsubsection{Relations}

		% There are five types of relations which define different types of relationships between activities in a workflow. 
		% These relations are:

		% \begin{description}
		% 	\item[Condition] If there is a condition relation from activity $A$ to activity $B$, then $B$ can only be executed if $A$'s executed attribute is true or $A$ is excluded.
		% 	\item[Response] If there is a response relation from activity $A$ to activity $B$, then $B$'s pending attribute will be set to true every time $A$ is executed.
		% 	\item[Include] If there is an include relation from activity $A$ to activity $B$, then $B$'s included attribute will be set to true every time $A$ is executed.
		% 	\item[Exclude] If there is an exclude relation from activity $A$ to activity $B$, then $B$'s included attribute will be set to false every time $A$ is executed.
		% 	\item[Milestone] If there is a milestone relation from activity $A$ to activity $B$, then $B$ can only be executed if $A$'s pending attribute is false or $A$'s included attribute is false.
		% \end{description}

		\subsection{SGX}
		Intel SGX is a technology which allows users to define protected areas of memory, so called enclaves, and run code within these enclaves.
		Intel guarantees that any code run this way is unmolested by any process running outside of the enclave.
		If the enclave contains secrets which the user wishes to keep, but still does not want to trust external processes with, the enclave can be sealed on disk, essentially encrypting it for later use.
		This means that data can be stored and processed securely, with its security guaranteed by Intel, if done properly.
		An additional feature of Intel SGX is that the result of any code run using Intel SGX can be verified by other users, given the code and the output.

		This is all made possible by unique keys generated at manufacturing and stored inside the fuse array of the processor.
		Some of these keys are known by Intel for the system to be recognized when contacting Intel servers.
		The only key which Intel allegedly do not possess is the key used for the sealing of enclaves, meaning that even Intel would theoretically not have access to sealed enclaves.

		In short, the major innovation in Intel SGX is the option of using asymmetric encryption with the keys already being distributed and tamper proof at hardware level.
		To ensure that keys are only distributed to trusted users, Intel offers Attestation Services (IAS) as a way of key distribution.

      \subsubsection{Enclave}
      What allows SGX enabled CPUs to provide these strong guarantees builds on the following two hardware details:
      \begin{description}
        \item Processor Reserved Memory (PRM): a sequential block of memory reserved for SGX inaccessible from untrusted software and even hardware.
        \item Enclave mode: a mode under which a logical processor gains access to the PRM, allowing trusted software exclusive access to its own memory.
      \end{description}
      Using these hardware facilities SGX defines the concept of an enclave.
      An enclave is, as its name suggests, a software module isolated completely from the rest of the system.
      Its memory is located solely in PRM, protecting it from tampering from other processes and enclaves running on the system.
      % ...
      % how?

      While SGX provides powerful guarantees trough its enclave concept, it does not guarantee software correctness and will not protect against flawed software.
      Instead SGX encourages developers to isolate a minimal piece of their software, the Trusted Computing Base (TCB), in a trusted enclave environment, and keep the remainder as traditional system processes.
      % ref SDG
      By minimizing the size of the TCB, and thus the amount of software one must trust, common security principles indicate that the chance of security flaws decreases.
      % principle
      % ref
      In order for an isolated enclave to be useful, communication between trusted and untrusted software is enabled through \textit{ECALLs} and \textit{OCALLS}.
      This interface must be defined at compile-time, specifying an API of ECALLs for the enclave as well as any untrusted services needed as OCALLs.

      % Signature
      When built, an enclave module is a plain binary on the untrusted file system.
      As an enclave under such circumstances would be vulnerable to tampering before initialization, SGX enforces a strict signature policy.
      An enclave must include an \textit{Enclave Signature} containing
      \begin{itemize}
        \item a hash of the code and initialization data of the enclave,
        \item the author's public key
        \item and an enclave version/product number.
      \end{itemize}
      During enclave initialization a hardware check is performed, ensuring the Enclave Signature matches the enclave binary loaded from the file system.
      % ...

      % Sealing
      Another powerful tool of an SGX enclave is the ability to read and write data to an untrusted storage medium while ensuring confidentiality of its contents.
      Such capabilities are needed as the PRM is volatile and will not persist a shut down.
      In SGX this process is known as \textit{sealing} and allows encryption and decryption confidential to either the Enclave Signature, allowing only the current enclave version access, or the Enclave Signature without the version number, allowing the current enclave version as well as future updates versions access.
      % encryption algos?
      % distinguising between running enclave instances

			\subsubsection{Attestation}

			Attestation, within the Intel SGX platform, is the action of verifying the existence of specific enclaves.
			The applications of this are two-fold, in the case where enclaves need to communicate to enclaves running on external CPUs and in the case where there are multiple enclaves running locally on the same CPU.
			These two situations are handled by two different processes, aptly named remote attestation and local attestation, respectively.
			In the former case of remote attestation, there is a considerable amount of setup needed to be done once, beforehand.
			A signed certificated from a recognized certificate authority needs to be obtained by the Independent Service Vendor (ISV).
			The certificate needs to then be registered with Intel, who upon acceptance will provide a Service Provider ID (SPID).

			The specific details of the remote attestation is then up to the API exposed by the ISV, but is outlined in figure\footnote{Taken from \url{https://software.intel.com/en-us/articles/intel-software-guard-extensions-remote-attestation-end-to-end-example}}.
			The enclave wishing to perform remote attestation requests it from the ISV API and is challenged to prove its identity.
			The enclave responds with its extended Group ID (GID), supplied by SGX using the Enhanced Privacy ID (EPID).
			The extended GID is then checked for support and if it is, the ISV service responds with a success message.
			The enclave then retrieves the actual GID and sends that to the ISV service, who checks that key with IAS.
			The ISV service then sends a Quote to the enclave, which attests that specific enclave to the IAS and the IAS only.
			This exchange is encrypted used the Diffie-Hellman Key Exchange. 

			For testing purposes, Intel also allows self-signed keys after a certificate has been obtained, which will likely be useful in the future work of this project.

			In the latter case of local attestation, the report key can be used by both enclaves, since they run on the same system, to sign messages, thereby proving that they run on the same system.
			This is done through a Diffie-Hellmann Key Exchange utilizing the report key\footnote{The flow of this exchange is described in more detail here \url{https://software.intel.com/en-us/node/709011}.}.
			Further communication between these enclaves can be sent through a protected channel.

		\subsection{Consensus}
		Consensus is the problem of establishing agreement between two or more processes in a distributed setting. Formally consensus is defined as a protocol in which $N$ processes ($P_0, P_1, \dots, P_{n-1}$) starts in an undecided state. Each process will then propose a single value, $v_i$, from a set of values, $D$. The processes communicate their propositions to each other, and each decide on a decision-variable, $d_i$. They are now in a decided state. If the following requirements are met, the protocol solves the consensus problem:
		\begin{description}
			\item[Termination] Every process eventually sets its decision-variable. 
			\item[Agreement] The decision-variables of all correct processes are set to the same value.
			\item[Integrity] If every correct process proposes the same $v$, then all correct processes decide $d_i = v$.
		\end{description}

		In a famous result from 1985, known as the FLP impossibility result\cite{flp}, it was shown that consensus was impossible to solve in an asynchronous setting with as little as a single crashing process. The FLP impossibility result can, very simplistically, be expressed as the problem of distinguishing a failed process from a delayed message in an asynchronous system.\\

	\section{Implementation}
	In order to demonstrate that the proposed solution could work, we demonstrate the implementation of a monotonic counter in \cpp.

	\subsection{Trusted monotonic counter example}
	The implemented example is separated into two parts:
	\begin{description}
		\item [TMCtest] is the enclave code. It is able to seal and unseal data, so that it is protected from tampering outside the enclave. It can create, increment and read monotonic counters. Due to the nature of enclaves, it is not able to make any I/O calls, which is why it needs the untrusted part of the code for this.
		\item [Untrusted] is the untrusted part of the code. It essentially acts as a wrapper for the enclave, in that it is responsible for all I/O calls. It is able to take and parse commands from the command-line and call the appropriate enclave function. It is also able to print and read enclave-sealed monotonic counters.  
	\end{description}
	The interface between TMCtest and Untrusted are specified by the EDL file \texttt{TMCtest.edl} (snippet \ref{snippet:tmctest.edl}). The EDL specifies which functions the untrusted code can call in the enclave (ECALLs), in the \texttt{trusted} block. 
	It is also able to specify which functions the enclave can call in the untrusted code (OCALLs), in the \texttt{untrusted} block, but there are none of these present in the example:
	\begin{snippet}[!ht]
	\begin{lstlisting}[language=C++, numbers=none]
enclave {
	from "sgx_tae_service.edl" import *;
    trusted {
        /* define ECALLs here. */
        public uint32_t create_sealed_monotonic_counter([out, size=sealed_mc_size]
            uint8_t* sealed_mc_result, uint32_t sealed_mc_size );
		public uint32_t increment_monotonic_counter([in,out, size=sealed_mc_size]
            uint8_t* sealed_mc_result, uint32_t sealed_mc_size);
		public uint32_t read_sealed_monotonic_counter([in,out, size=sealed_mc_size]
            uint8_t* sealed_mc_result, uint32_t sealed_mc_size, [in, out] uint32_t* mc_value);
		public uint32_t get_size();
    };

    untrusted {
        /* define OCALLs here. */
    };
};
	\end{lstlisting}
	\caption{\texttt{TMCtest.edl}, the interface between the enclave and the untrusted code\label{snippet:tmctest.edl}}
	\end{snippet}

	The struct created for the monotonic counter data is called \texttt{monotonic\_counter} (snippet \ref{snippet:monotonic-counter}), and is only accessible for the enclave. It consists of the current monotonic counter value, as wells as a unique id for the monotonic counter, which SGX uses for accessing the appropriate non-volatile memory.
	\begin{snippet}[!ht]
		\begin{lstlisting}[language=C++, numbers=none]
typedef struct _monotonic_counter
{
  sgx_mc_uuid_t mc;
  uint32_t mc_value;
}monotonic_counter;
		\end{lstlisting}
		\caption{monotonic\_counter struct in \texttt{TCMtest$.$cpp} \label{snippet:monotonic-counter}}
	\end{snippet}

	When creating an accessing monotonic counters, SGX needs to establish a session with the so-called Platform Services Enclave (PSE), which is a protected enclave, only accessible by other enclaves. The PSE provides access to protected functionality, such as monotonic counters, sealing and attestation.
	After establishing this session, the creation of a monotonic counter is simply a call to \texttt{sgx\_create\_monotonic\_counter}.
	When the monotonic counter has been created, the struct is sealed and passed to the untrusted code. Here it is important to notice that the memory containing the unsealed data must be cleared before returning the sealed data to the untrusted code, in order to prevent leaking data.

	Incrementing a monotonic counter is very similar to creating it. Due to the fact that the untrusted code only has access to sealed structs, however, the passed data must first be unsealed. This is done very simply with a call to \texttt{sgx\_unseal\_data}. The data is also verified (i.e. that the value of the monotonic counter is as expected), to ensure that no tampering has taken place. The verification does not guarantee that another instance of the enclave has not unsealed the data and incremented the counter. However, it has to be another instance of the exact enclave, as the sealing process is encryption with a Seal Key, which is unique to the enclave and platform \cite{sgx-dev-guide}.


	\section{Challenges}

		\subsection{SGX security}

		Due to SGX being a security guarantee made by Intel all work involving it, the proposed project requires absolute trust in Intel keeping private keys secure and them not violating the confidentiality of machines with their hardware.

		\subsection{Consensus}
		Given that our problem deals with agreeing on a history of a workflow, we must somehow circumvent FLP impossibility.
		A few solutions for circumventing FLP-impossibility have been suggested, for instance using a probabilistic protocol, which achieves consensus except for some negligible probability. Most notable of these is perhaps the blockchain technology, (formal analyses of the complexity and probability of success can be found in \cite{miller2014anonymous}). However, the basis of a blockchain succeeding is an incentive for peers to use computing power to secure the system. This incentive takes the form of a currency \cite{bitcoin-white-paper}. While a clever solution, we are not interested in basing our system on a currency, thus subjecting the viability of the system on the rise-and-fall of the rate and popularity of the currency. Instead we have looked at alternatives for side-stepping FLP-impossibility. We have found the survey by Correia et al. \cite{consensus-survey} to be very helpful. The following are the solutions that we have identified to have the most potential for this system:
		\begin{itemize}
			\item One solution is to assume that some subsystem or step of the protocol behaves synchronously. This is known as \textit{partial synchrony}. If the system does not live up to these assumptions, the protocol will naturally fail. The challenge is this solution to identify a subsystem that can reliably be assumed to behave synchronously. 
			\item A variation on \textit{partial synchrony} is to assume some subsystem has some "good" properties, which is otherwise not guaranteed in the environment. These subsystems are known as \textit{wormholes}. According to the survey \cite{consensus-survey}, the Trusted Platform Module (TPM) is an example of a wormhole, and since SGX can implement some of the same concepts as those available in TPM, it might be possible to use SGX as a wormhole. The survey mentions several protocols which might be transformed to solve the consensus problem, including one that uses a Unique Sequential Identifier Generator, which is essentially a monotonic counter. The challenge in this approach is to identify which of the properties in the wormhole allows side-stepping FLP impossibility, and then implementing them using SGX.
			\item The most simple way of side-stepping FLP impossibility, is simply to modify the consensus problem, i.e. weaken the requirements s.t. some failures are allowed. For instance, the PAXOS algorithm \cite{paxos} satisfy variations on the Agreement and Integrity requirements, but does not satisfy the appertaining Termination requirement \cite{fast-byz-cons}. The challenge in this approach is to identify which requirements can be most safely weakened, and how.
		\end{itemize}
		
	\section{Future Work}

	\begin{thebibliography}{9}

		\bibitem{smart-contracts}
		Nick Szabo.
		\textit{Formalizing and Securing Relationships on Public Networks}.
		Peer-reviewed journal on the internet.

		\bibitem{ethereum-white-paper}
		\textit{Ethereum White Paper}.
		\url{https://github.com/ethereum/wiki/wiki/White-Paper}.
		Accessed 2017-12-11.

		\bibitem{bitcoin-white-paper}
		\textit{Bitcoin White Paper}.
		\url{https://bitcoin.org/bitcoin.pdf}.
		Accessed 2017-12-11.

		\bibitem{dcr-paper}
		Thomas T. Hildebrandt \& Raghava Rao Mukkamala.
		\textit{Declarative Event-Based Workflow as Distributed Dynamic Condition Response Graphs}.
		IT University of Copenhagen.

		\bibitem{intel-sgx-explained}
		Victor Costan \& Srinivas Devadas.
		\textit{Intel SGX Explained}.
		Computer Science and Artificial Intelligence Laboratory Massachusetts Institute of Technology.

		\bibitem{dcreum}
		Mikkel Gaub, Tróndur Høgnason, Malthe Ettrup Kirkbro \& Mads Frederik Madsen.
		\textit{Consensus in declarative process models using distributed smart-contracts}.

		\bibitem{sgx-dev-ref}		
		\textit{Intel® Software Guard Extensions SDK for Linux OS}.
		\url{https://download.01.org/intel-sgx/linux-2.0/docs/Intel_SGX_SDK_Developer_Reference_Linux_2.0_Open_Source.pdf}

		\bibitem{sgx-dev-guide}
		\textit{Intel® Software Guard Extensions Developer Guide}
		\url{https://software.intel.com/en-us/documentation/sgx-developer-guide}

		\bibitem{flp}
		Michael J. Fischer and  Nancy A. Lynch and Michael S. Paterson
		\textit{Impossibility of distributed consensus with one faulty process},
		\url{https://dl.acm.org/citation.cfm?id=214121}

		\bibitem{miller2014anonymous},
		Andrew Miller and Joseph J. LaViola, Jr.
  		\textit{Anonymous byzantine consensus from moderately-hard puzzles: A model for bitcoin},
		\url{Available on line: http://nakamotoinstitute.org/research/anonymous-byzantine-consensus}

		\bibitem{consensus-survey}
		Miguel Correia, Giuliana Santos Veronese, Nuno Ferreira Neves and Paulo Verissimo
		\textit{Byzantine consensus in asynchronous message-passing systems: a survey} 

		\bibitem{paxos}
		Leslie Lamport
		\textit{The Part-Time Parliament} 

		\bibitem{fast-byz-cons}
		Jean-Philippe Martin and Lorenzo Alvisi
		\textit{Fast Byzantine Consensus}



	\end{thebibliography}

\end{document}
